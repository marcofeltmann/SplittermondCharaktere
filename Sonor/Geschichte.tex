\documentclass{report}

\usepackage[ngerman]{babel}
\usepackage[utf8]{inputenc}
\usepackage[T1]{fontenc}

\usepackage[babel,german=swiss]{csquotes}

\author{Marco Feltmann}
\title{Hintergrundgeschichte des Charakters \\ \enquote{Sonor} }

\begin{document}

	\maketitle
	
	\tableofcontents

	\chapter{Prolog}
Alle Sandrasselottern besitzen ein hochpotentes Schlangengift. Das Gift enthält
unter anderem ein hochwirksames Hämotoxin (Blutgift) und ein weniger wirksames
Neurotoxin (Nervengift). Für die Störung der Blutgerinnung ist das Enzym Ecarin
verantwortlich. Das Gift ist im Kreislaufsystem sehr stabil, sodass die
Ungerinnbarkeit des Blutes über Wochen hin anhalten kann. Nach einem Biss kommt
es innerhalb von ein bis sechs Stunden zu unstillbaren Blutungen aus der
Bisswunde sowie über die Schleimhäute, wodurch Blut aus Nase, Mund und Darm
austritt. Die Haut färbt sich um die Bissstelle herum. Das gebissene Glied
schwillt extrem an und es entstehen Nekrosen. Weitere Folgen können
Bluterbrechen, blutiger Speichel und Blutergüsse unter der Haut, sowie
Herzrhythmusstörungen, Blutdruckabfall, Hirnblutungen und Nierenschäden sein. Es
können auch Lähmungserscheinungen und ein Schockzustand auftreten. Ohne
Antiserumbehandlung kommt es in den meisten Fällen zum Tode.

Gegen das Gift der Ägyptischen Sandrasselotter gibt es kein spezielles Antiserum
(Antivenin), bei einem Biss wird entsprechend ein allgemein bei Echis-Arten
nutzbares Mittel eingesetzt.

\chapter[Datum erstes Spiel]{Das erste Abenteuer}
\section{Die Bitte des Geistältesten}

\section{Seltsame Ereignisse}
\subsection{Albin}
\subsection{Mensch}

\section{Ärger in der Stadt}

\section{Eine neue Aufgabe}

\section{Erste Ermittlungen}


\chapter[Datum zweites Spiel]{Das zweite Abenteuer}

\end{document}
