\begin{center}
\enquote{\emph{Ioria} -- der Nabel der Welt. Die Orakelstadt. Der Sündenpfuhl.}
\end{center}

\section{Seltsame Resonanzen}
Wenige Schritte nachdem Sonor den Hafen verlassen hat und sich interessiert umschaut, passiert etwas sehr merkwürdiges. Die Welt um ihn herum scheint für den Bruchteil einer Sekunde völlig still zu stehen. Alle Farbe scheint aus den umgebenden Wesen heraus gewaschen und sie verschwimmen fast vor seinen Augen. Er hört eine schwache Melodie klingen.
Lediglich eine dunkel gekleidete Gestalt mit unter einer Kapuze verborgenem Gesicht sticht scharf aus der Umgebung hervor und scheint vom Stillstand nicht betroffen.

Kaum ist dieses seltsame Gefühl vorüber, wird Sonor von hinten angerempelt. Offenbar war ein drittes Wesen an dieser Erfahrung beteiligt, welches er allerdings nicht sehen konnte, da sie hinter ihm stand.

\subsection{Lian, feurige Albin des Phoenixordens}
Die Gestalt, die den Varg angerempelt hatte, ist nur etwa 30 Zentimeter kleiner als er. Auffällig an ihr ist\ldots Nun, eigentlich alles. Blasse Haut, rotbraunes Haar, rote Augen, roter Schuppenpanzer, eine rote Insignie mit Ähnlichkeit eines brennenden Vogels. Weiterhin trägt sie ein Schwert mit sich herum.

Nach eigenen Angaben möchte sie den hiesigen Orden ihrer Religion besuchen. Irgendwas mit Fennex oder so. 

\subsection{Volton, alchemiekundiger Mensch}

Der auffallend kleine Mensch wirkt sehr ausweichend und scheint sich in seinem Kapuzenmantel zu verstecken. Eventuell möchte er seine seltsamen unterschiedlichen Augen verbergen, eines weiß, das andere rot. Insgesamt ist er mit Rucksack und Stab ziemlich bepackt und wird von einer schwarzen Krähe begleitet.

\section{Eifriger Eber}

Nach einer kurzen Rücksprache mit den Geistern der Ahnen folgt Sonor den beiden Fremden in die Stadt.


Die Gruppe kehrt im Gasthof \enquote{Eifriger Eber} ein, worauf Volton die Aufmerksamkeit aller Anwesenden auf sich zieht. Zu Sonors Überraschung wird die auffallend rote Lian nur ganz kurz beäugt und er selbst überhaupt nicht wahrgenommen.

Der übertrieben freundliche Gastwirt nimmt die Bestellung der Gruppe auf und die drei nutzen die Wartezeit einander vorzustellen.

Lian und Sonor haben das gleiche Ziel: Das Orakel. Volton beschließt die Beiden zu begleiten, um das Orakel nach dem jüngsten Ereignis zu befragen.
Lian sieht dem längsten Aufenthalt entgegen, da sie noch Aufgaben für ihren Orden in Ioria erledigen möchte.
Sonor stuft sie als eine Art Geistheilerin ein, die anstatt mit ihren Ahnen mit ihrem Fönicks spricht. 

Nach dem Essen begeben sich die drei zur Ruhe, nachdem sie der vehementen Einladung der letzten drei Gäste nicht nachkommen.
Sonor lässt sich im Gemeinschaftsschlafsaal nieder während Volton und Lian jeweils ein Einzelzimmer beziehen.
Für Sonor gestaltet sich die Wahl des Gemeinschaftszimmers als ausgesprochen nachteilig, da sich zu später Stunde vier grölende und lärmende stark angetrunkene Gäste ebenfalls im Gemeinschaftssaal niederlassen.

\section{Ärger in der Stadt}

Obwohl der Lärm, den diese Kerle produzierten, Sonor den erholsamen Schlaf versagten, fühlt er sich am nächsten Morgen nicht übermäßig gerädert.

Volton und Lian sitzen bereits unten und haben sich von der Wirtin Frühstück bringen lassen. Die Wirtin scheint Lian sehr zugetan zu sein. Sonor verabschiedet sich aus der Ferne, da er früh zum Orakel aufbrechen will. Gerade im Begriff die Tür zu öffnen rempelt ihn ein Mensch in weißer Uniform fast um, als dieser hinein stürmt und Volton wegen Kindesentführung als verhaftet bezeichnet.
Einer Eingebung folgend schließt Sonor die Tür von innen und versperrt diese.

Lian scheint unerfreut über diese Unterbrechung und dem Varg fällt auf, dass um ihre Hand rote Funken auflodern.
Nach längerem Hin und Her, in der sich der Weiße mal im Nachteil, mal im Vorteil wähnt, lässt Sonor dann den Rest der Truppe in die Taverne.
Ein offenbar ranghöherer Weißer erklärt ihnen, dass lediglich ein dem Protokoll entsprechendes Verhör statt finden wird, da man Volton die Vorgehensweise der Zutrittsbeschaffung nicht zutraut.

Von den vier anderen Gästen fehlt jede Spur, die Wirtin beteuert allein mit ihrer Tochter den Eber zu führen und stellt fest, dass auch ihre Tochter offenbar verschwunden ist.

Nachdem Volton noch irgend einen Brief an irgend einen Mentor geschrieben hat, begeben sich die drei mit auf die Wache.

Das Abgeben der Waffen am Empfang scheint für Lian ein regelrechtes Problem zu sein, was Sonor überhaupt nicht nachvollziehen kann. Erst als ein älterer Herr mit Schildkröteninsignie in den Raum tritt und ihr versichert die Waffe wie seine Eigene zu hüten, entspannt sie sich sichtlich. Das heißt, die Funken um ihre Hand und das Leuchten ihrer roten Augen wird weniger.

Das Verhör wird von einem Vaigarr durchgeführt und ist entsprechend informativ. Nix gesehen, nix gehört, Opfer unbekannt, Tschüss.
Offensichtlich kamen die beiden Anderen nicht so glimpflich davon, da Sonor sehr lange auf die Rückkehr der beiden warten musste.
Ihnen allen wurde mitgeteilt, dass sie bis zur Aufklärung dieser Entführung die Stadt nicht verlassen dürfen.

Der Tag ist bereits voran geschritten und sie stehen nahe am Orakel, so dass Lian sich überreden lässt zunächst das Orakel und erst danach ihren Orden aufzusuchen.

Besagtes Orakel beantwortet die Frage des Ältesten Geistheilers mit einem seltsamen Singsang:
\enquote{Die Schatten heben und senken sich. Die Wüste fürchtet sich. Die Tarr werden bestehen. Die Wüste wird leben.}


\section{Eine neue Aufgabe}

Im Anschluss spricht noch irgend ein Priester mit so einem Schildkrötenemblem mit den Dreien, erzählt irgend eine zusammenhanglose Prophezeiung, sinniert darüber, dass sie zu fünft sein sollten und allerhand ähnlich seltsames Zeug.
Auf Voltons Frage, was das am vorherigen Tag war, benutzte der Mann das Wort \enquote{Splitterträger}. 
Auch er wies darauf hin, dass es sinnvoll sei zunächst in Ioria zu verweilen.

Beim Phönixorden gehen die rätselhaften Ereignisse weiter. Die spartanische Einrichtung und Bewirtung sind das Einzige, mit dem Sonor irgend etwas anfangen kann. Das Gespräch zwischen Liam und einer anderen in noch kräftigerem Rot gekleideten Frau findet in einer Sprache statt\ldots Genau genommen hört es sich für Sonor überhaupt nicht nach einer Sprache an.

Lian klärt sie auf, dass in der Unterstadt Iorias in letzter Zeit viele Kinder entführt wurden. Der Orden beauftragte sie damit, die Täter aufzuspüren und vorzuführen, da sich die seltsamen weiß gekleideten Sicherheitskräfte für die Unterstadt wohl nicht zuständig fühlen.
Weiterhin erzählt sie, dass sie die Namen der jüngsten Opfer bekommen hat und sofort mit der Untersuchung beginnen möchte.

Sonor und Volton willigen ein ihr zu helfen.