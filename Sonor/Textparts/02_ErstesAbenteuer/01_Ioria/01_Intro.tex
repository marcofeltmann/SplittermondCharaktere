Die Reise mit dem Schiff von Surmakar bis Ioria dauert dem jungen Vargen viel zu lange. 
Vor Allem stellt sich die Fahrt für ihn, der nur Sandmeere und Dünen kennt, als sehr eintönig und unliebsam heraus.
Das Auf und Ab des Schiffes in den Wellen erinnert ihn ein wenig an das Geschaukel auf den Schwarzkamelen, die er als junger Welpe geritten war, bevor das Missgeschick mit seinen Klauen passierte.
Allerdings brachte die Unruhe der Kamelreise ihn von allein auf den Gedanken, dass die Fortbewegung zu Fuß die einzige von der Natur vorgegebene Art und Weise des Reisens sei.
Diese Schifffahrt festigt seine Meinung diesbezüglich noch weiter.
 
Nach einer gefühlten Ewigkeit kann Sonor endlich den Hafen Iorias sehen.
Gespannt wartet er auf die Ankunft des Schiffes und brennt voller Eifer darauf, den Auftrag auszuführen und das Orakel wegen der Zeichen der Wüste zu befragen, zu denen sich die Geister der Ahnen so beharrlich ausschwiegen. Selbstverständlich brennt auch sein eigenes Ziel in ihm und er hofft auf einen Fingerzeig zum Lebensraum seines Geisttieres.

Vom Hafen ist es noch ein Stück in die Stadt. Der Boden sagt seinen Füßen überhaupt nicht zu, er ist viel zu hart und viel zu kalt. 
Wobei die Behauptung \enquote{zu kalt} für das gesamte Areal hier zutrifft.
Interessiert blickt er sich auf dem Weg in die Stadt um. Kreaturen unterschiedlichster Art und Abstammung tummeln sich auf den Straßen, ihre Kleidung so verschieden wie ihre Herkunft. Ihm wird klar, dass er sich hier an einem zentralen Punkt befindet, der die reisenden Händler zu ihren Erzählungen anregt und die unterschiedlichen Wesen Lorakis' hier zusammen führt. 