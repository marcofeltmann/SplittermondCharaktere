\documentclass{article}

\usepackage[ngerman]{babel}
\usepackage[utf8]{inputenc}
\usepackage[T1]{fontenc}

\usepackage[babel,german=swiss]{csquotes}

\author{Marco Feltmann}
\title{Hilfreiche Fragen für die Ausgestaltung \\ \enquote{Sonor} }

\begin{document}

	\maketitle
	
	\tableofcontents
	
	\section[Aussehen]{Wie sieht der Abenteurer aus, wie wirkt er auf Fremde?}
	
	Sonor ist ein Varg mit sandfarbenem Fell. Trotz
	seiner 2,27 Metern Größe und über 150kg Körpergewicht wirkt er
	verhältnismäßig schmal. Sein Gesicht hat schmale lange Linien, während
	seine dunkelbraunen Augen und seine Ohren unverhältnismäßig groß sind.
	Tatsächlich besteht sein Gesicht zu knapp der Hälfte aus Ohren, so dass
	die verhältnismäßig kleine, kurze und schmale Schnauze kaum auffällt.

	Auch ist sein Gebiss eher zierlich für einen Varg. Zwar sind die Zähne
	scharf und spitz wie bei anderen Vertretern, allerdings hat er
	weniger und sie sind ziemlich kurz.

	Gekleidet ist er in braune Wollstoffe mit dunklen Lederapplikationen.
	Auffällig sind die riesigen Waffen, die er mit sich herumträgt. Sowohl
	der Glenn als auch die Speerschleuder sind jeweils über 2 Meter lang und
	eindrucksvoll, da Sonor für einen Varg eher schmächtig wirkt.

	Vor Allem abends und in der Nacht fällt dem geneigten Beobachter auf,
	dass aus einem Lederbeutel an seiner Hose ein kleiner, kurzer,
	Kopf herauslugt, der ein wenig wie ein abgerundetes Dreieck aussieht.
	Er gehört zu der knapp 35cm kurzen und schlanken Begleiterin Wadjet,
	deren rotbraunes Schuppenkleid durch einige sandfarbene Flecken
	unterbrochen wird, welche Sonors Fell ähneln.
	Vor Allem vor ihren Ruhezeiten ertönt ein seltsames Gerassel aus dem
	Beutel, das durch die übereinander gleitenden schräg angeordneten und
	gekielten Seitenschuppen Wadjets verursacht wird.

	Sonor selbst wirkt ob dieses bedrohlichen und giftigen Bewohners seines
	Beutels in keinster Weise besorgt. Auf die Frage, ob er keine Angst vor
	Bissen habe, antwortet er nahezu gleichgültig, dass sie das schon oft
	gemacht hat. Es gäbe also nichts zu befürchten\ldots

	
	\section[Soziales Umfeld]{In welchem Umfeld ist der Abenteurer aufgewachsen?}
	
	Sonor stammt aus der Wüste Surmakar, der 'Sonnenweite' und ist das
	einzige Kind einer Familie von Schwarzkamelzüchtern. Bei jedem Zug durch
	die Wüste durfte seine Familie in direkter Nähe von Oasen oder
	Wasserlöchern hausen, da die Kamele ja mit Wasser versorgt werden
	mussten - die Kamelversorgung zählte zu seinen Hauptaufgaben im Stamm.

	Die Oasen waren auch immer die heiligen Stätten des Stammes, da ihre
	Entstehung und Existenz für das Leben in der Sonnenweite essentiell ist,
	ihr Austrocknen dagegen ein untrügliches Zeichen des Vergehens ist.
	
	Sowohl Geburten wie Lebensbundschließungen als auch Einäscherungen wurden immer
	an diesem heiligen Ort in zeremonienartigen Ritualen durchgeführt.

	\section[Beziehungsstatus]{Hat der Abenteurer eine eigene Familie oder eine große Liebe?}

	\section[Freund und Feind]{Hat der Abenteurer einen besten Freund und/oder einen ärgsten Feind?}

	\section[Religion]{Zu welchen Göttern betet der Abenteurer?}
	
	\section[Magie]{Wie geht der Abenteurer mit Magie um?}

	\section[Schicksal]{Wie kam der Abenteurer zum Abenteurerleben?}

	\section[Antrieb]{Was ist der sehnlichste Wunsch des Abenteurers?}

	\section[Ängste]{Wovor fürchtet sich der Abenteurer am Meisten?}

	\section[Moral]{Wie sehen die Moralvorstellungen des Abenteurers aus?}
	
	\section[Frustrationstoleranz]{Wie steht der Abenteurer zu Gewalt und dem Töten von Lebewesen?}

	\section[Macken]{Pflegt der Abenteurer seltsame Verhaltensweisen oder Macken?}
	
	\section[Humor]{Versteht der Abenteurer Spaß?}
	
	\section[Aufgeschlossenheit]{Ist der Abenteurer aufgeschlossen gegenüber Neuem?}
	
	\section[Top Secret]{Was ist das tiefste Geheimnis des Abenteurers?}
	
	\section[Splitter]{Weiß der Abenteurer, dass er ein Splitterträger ist, und wie beeinflusst ihn das?}

\end{document}
