\documentclass{article}

\usepackage[ngerman]{babel}
\usepackage[utf8]{inputenc}
\usepackage[T1]{fontenc}

\usepackage[babel,german=swiss]{csquotes}

\author{Marco Feltmann}
\title{Hilfreiche Fragen für die Ausgestaltung \\ \enquote{Sonor} }

\begin{document}

	\maketitle
	
	\tableofcontents
	
	\section[Aussehen]{Wie sieht der Abenteurer aus, wie wirkt er auf Fremde?}
	
	Sonor ist ein Varg mit sandfarbenem Fell. Trotz
	seiner 2,27 Metern Größe wirkt er mit seinen knapp 125kg Körpergewicht
	verhältnismäßig schmal. Sein Gesicht hat schmale lange Linien, während
	seine dunkelbraunen Augen und seine Ohren unverhältnismäßig groß sind.
	Tatsächlich besteht sein Gesicht zu knapp der Hälfte aus Ohren, so dass
	die verhältnismäßig kleine, kurze und schmale Schnauze kaum auffällt.

	Auch ist sein Gebiss eher zierlich für einen Varg. Zwar sind die Zähne
	scharf und spitz wie bei anderen Vertretern, allerdings hat er
	weniger und sie sind ziemlich kurz.
	Seine Krallen hingegen sind ziemlich beachtlich und gut geschärft. Er
	kann sie im Gegensatz zu den meisten anderen Varg nicht einziehen, da
	sie fest an den Knochen verwachsen sind, was sie allerdings
	widerstandsfähiger macht als bei anderen Varg.

	Gekleidet ist er in braune Wollstoffe mit dunklen Lederapplikationen.
	Auffällig sind die riesigen Waffen, die er mit sich herumträgt. Sowohl
	die Glefe als auch die Speerschleuder sind jeweils über 2 Meter lang und
	umso eindrucksvoller, da Sonor für einen Varg eher schmächtig wirkt.

	Vor Allem abends und in der Nacht fällt dem geneigten Beobachter auf,
	dass aus einem Lederbeutel an seiner Hose ein kleiner, kurzer,
	Kopf herauslugt, der ein wenig wie ein abgerundetes Dreieck aussieht.
	Er gehört zu der knapp 35cm kurzen und schlanken Begleiterin Wadjet,
	deren rotbraunes Schuppenkleid durch einige sandfarbene Flecken
	unterbrochen wird, welche Sonors hauptsächlicher Fellfarbe ähneln.
	Vor Allem vor ihren Ruhezeiten ertönt ein seltsames Gerassel aus dem
	Beutel, das durch die übereinander gleitenden schräg angeordneten und
	gekielten Seitenschuppen Wadjets verursacht wird.

	Sonor selbst wirkt ob dieses bedrohlichen und giftigen Bewohners seines
	Beutels in keinster Weise besorgt. Auf die Frage, ob er keine Angst vor
	Bissen habe, antwortet er nahezu gleichgültig, dass sie das schon oft
	gemacht hat. Es gäbe also nichts zu befürchten\ldots

	Bei spirituellen Anlässen wie der gemeinsamen Jagd mit seinem Stamm
	wählt er immer Kreidestaub und Kohle aus, um sich sein Antlitz im Stile
	seines Schädels zu bemahlen: Der Bereich um die Augen, Lefzen und Ohren schwarz
	(die Nasenspitze ist es ja schon), der Rest weiß mit angedeuteten Zähnen
	auf den Lefzen und zusätzlichen schwarzen Verziehrungen um die Augen und
	Nase.
	
	\section[Soziales Umfeld]{In welchem Umfeld ist der Abenteurer aufgewachsen?}
	
	Sonor stammt aus der Wüste Surmakar, der 'Sonnenweite' und ist das
	einzige Kind einer Familie von Schwarzkamelzüchtern. Bei jedem Zug durch
	die Wüste durfte seine Familie in direkter Nähe von Oasen oder
	Wasserlöchern hausen, da die Kamele ja mit Wasser versorgt werden
	mussten - die Kamelversorgung zählte zu seinen Hauptaufgaben im Stamm.

	Die Oasen waren auch immer die heiligen Stätten des Stammes, da ihre
	Entstehung und Existenz für das Leben in der Sonnenweite essentiell ist,
	ihr Austrocknen dagegen ein untrügliches Zeichen des Vergehens ist.
	
	Sowohl Geburten wie Lebensbundschließungen als auch Einäscherungen wurden immer
	an diesem heiligen Ort in zeremonienartigen Ritualen durchgeführt.


	\section[Beziehungsstatus]{Hat der Abenteurer eine eigene Familie oder eine große Liebe?}

	Sonor hat weder Geschwister noch eine Partnerin. Lediglich seine Eltern
	in Surmakar und sein Stamm sind in seinem Leben wichtig.


	\section[Freund und Feind]{Hat der Abenteurer einen besten Freund und/oder einen ärgsten Feind?}

	Nein.


	\section[Religion]{Zu welchen Göttern betet der Abenteurer?}
	
	Die Religion Sonors Stammes ist kompliziert. Im Prinzip beten sie die
	Geister ihrer Ahnen an und führen ihre Rituale an ihrer Meinung nach magischen
	Orten durch.

	Sonor selbst scheint manchmal Fragen zu murmeln und angestrengt auf
	Antworten zu warten. Ob das seine Art von Gebet ist oder er einfach nur
	introvertiert ist, ist für Dritte schwer zu unterscheiden.


	\section[Magie]{Wie geht der Abenteurer mit Magie um?}

	Sonor führt die Gepflogenheiten seines Stammes fort. Magie wird in
	Ritualen zur Kampfvorbereitung genutzt, um Ausrüstungen zu verbessern,
	die eigenen Kampfreflexe zu stärken oder die erlegten Tiere haltbarer zu
	machen.

	Es ist allerdings sehr verpönt, bei der Jagd direkt mit Elementar- oder
	ähnlicher Magie zu hantieren. Dies erzürne die Ahnen, die den Akt der Jagd an
	sich als Handwerk ansehen, der nur und ausschließlich durch die
	physischen und psychischen Fähigkeiten des Varg vollzogen werden darf.
	Nachweisbare Zuwiderhandlungen werden mit Ausschluss aus dem Stamm geahndet.


	\section[Schicksal]{Wie kam der Abenteurer zum Abenteurerleben?}

	Bei den Sandläufern der Tarr ist es seit je her Tradition, dass der
	älteste Geistheiler des Stammes sie aus ihrem Lehrlingsstand enthoben
	hat. Wie viele vor ihm hat Sonor die Wüste auf der Suche nach dem Wesen
	durchstreift, das seinen Geist erwachsen machen sollte. 

	Nach vielen Jagden, auf deren auch Wadjet ihm begegnete und ihn seither
	beim Erkunden winziger Höhlen und Gänge unterstützt, musste er entmutigt
	feststellen, dass sein Geisttier sich außerhalb der Wüste befindet.

	Sein Mentor gab ihm den Hinweis auf den Weg, dass es die Ahnen erzürne
	Magie zum Aufspüren des Wesens einzusetzen denn das Tier kommt zum Tarr,
	wenn es ihn für würdig hält.

	Die Worte seines Mentoren auf die Goldwaage legend beschloß Sonor, das
	Orakel von Ioria nach dem Gebiet zu befragen, in dieses Tier sich
	aufhält. Schließlich möchte Sonor es ihm so einfach wie möglich machen
	ihn zu finden.


	\section[Antrieb]{Was ist der sehnlichste Wunsch des Abenteurers?}

	Sonor möchte endlich seinen Lehrlingsstatus hinter sich bringen, um
	seinen Stamm sicher durch die Sumarkar leiten und sie mit Nahrung
	versorgen zu können.

	
	\section[Ängste]{Wovor fürchtet sich der Abenteurer am Meisten?}

	Seine größte Angst ist es, das Tier welches seinen Geist erwachsen
	machen soll nicht aufzuspüren. Jeden Tag bittet er seine Ahnen,
	schützend ihre Hände über jenes Wesen zu legen, damit es nicht durch
	unwürdige Lebensformen oder widrige Umstände den Tot findet bevor Sonor
	es gefunden hat.


	\section[Moral]{Wie sehen die Moralvorstellungen des Abenteurers aus?}
	

	Sonor hat kurz nach seinem Aufbruch aus der Surmakar festgestellt, dass
	die Moralvorstellungen seines Stammes nicht unbedingt dem entspricht,
	das andere Völker ausleben. 

	Er ehrt das Leben an sich und als solches, was immer er bekommt teilt er
	unter den Stammesmitgliedern auf, er respektiert das Eigentum Anderer
	und setzt sich nach seinen Kräften für das Wohl seines Stammes ein.


	\section[Frustrationstoleranz]{Wie steht der Abenteurer zu Gewalt und dem Töten von Lebewesen?}


	Leben nehmen um Leben zu erhalten ist in Sonors Welt der einzig richtige
	Umgang mit diesem einmaligen Naturschauspiel. Dies spiegelt sich auch
	den den Jagdritualen wider, in denen nach erfolgreichem Beuteschlag den
	Geistern der Ahnen zu Ehren und Dank ein Fest abgehalten wird.

	Doch auch im Verteidigungsfall, sollte die Auseinandersetzung dem
	Gegenüber das Leben kosten, wird der Leichnam fürsorglich der Obhut der
	Geister der Ahnen übergeben.

	Körperliche Auseinandersetzungen an sich sind allerdings etwas ganz
	Anderes. Sie gehören zum Leben dazu wie das Essen und das Schlafen.
	Wer sich im Recht fühlt und nicht darum zu kämpfen versteht, ist als
	Varg nicht ernst zu nehmen. Wer einer Auseinandersetzung entflieht,
	ist sich auch seiner Worte nicht sicher und demnach ein Aufschneider.

	Allerdings musste Sonor auf seinen Reisen feststellen, dass das Leben
	außerhalb der Surmakar ein Anderes ist. Die letzte körperliche
	Auseinandersetzung mit einem gnomischen Händler, der ihm überteuerte
	und welke Heilkräuter verkaufen wollte, zeigte Sonor deutlich, dass
	dünne Haut seinen Pranken nicht so gut Stand hält wie festes Tarrfell.
	Die Tatsache, dass er zu den wenigen Varg gehört, die /textbf{unfähig}
	sind ihre Klauen einzufahren, lässt ihn Auseinandersetzungen auf sich zu
	kommen statt ihnen entgegenzurennen.


	\section[Macken]{Pflegt der Abenteurer seltsame Verhaltensweisen oder Macken?}



	
	Sonor wirkt einerseits sehr stoisch und mitunter phlegmatisch.
	Mitunter hat man das Gefühl, er führe Selbstgespräche und warte
	angestrengt auf eine Antwort.

	Ist er sich allerdings einer Sache sehr sicher, kommt der Varg in ihm
	durch. Blitzartige, schnelle und kraftvolle Bewegungen lassen einen ganz
	anderen Sonor zum Vorschein kommen.
	

	\section[Humor]{Versteht der Abenteurer Spaß?}
	




	\section[Aufgeschlossenheit]{Ist der Abenteurer aufgeschlossen gegenüber Neuem?}
	
	\section[Top Secret]{Was ist das tiefste Geheimnis des Abenteurers?}
	
	\section[Splitter]{Weiß der Abenteurer, dass er ein Splitterträger ist, und wie beeinflusst ihn das?}

\end{document}
